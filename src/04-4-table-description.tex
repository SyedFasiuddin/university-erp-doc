\subsection{Table Description}

\subsubsection{Departments}

There are various departments in a university and this table is used to keep
record of all those departments with their short codes and ID's of respective
HOD's.

\begin{lstlisting}[basicstyle=\small]
     Column      |          Type          | Nullable
-----------------+------------------------+----------
 full_name       | character varying(100) | not null
 short_code      | character varying(5)   | not null
 hod_lecturer_id | character varying(15)  |
\end{lstlisting}

\subsubsection{Students}

The students table is used to stores the detail of every students enrolled in
various departments across the university, details such as their name, fathers
name, phone number, their previous academic details etc are stored in this
table.

\begin{lstlisting}[basicstyle=\small]
       Column        |          Type          | Nullable
---------------------+------------------------+----------
 usn                 | character varying(15)  | not null
 name                | character varying(30)  |
 fathers_name        | character varying(30)  |
 mothers_name        | character varying(30)  |
 dob                 | date                   |
 address             | character varying(255) |
 phone_num           | numeric(10,0)          |
 email               | character varying(30)  |
 department          | character varying(5)   | not null
 last_qualification  | character varying(50)  |
 qualified_from      | character varying(255) |
 passing_year        | date                   |
 marks_scored        | numeric(3,0)           |
 percentage          | numeric(4,2)           |
 cet_rank            | integer                |
 neet_rank           | integer                |
 aadhar_number       | numeric(12,0)          |
 bank_account_number | numeric(12,0)          |
\end{lstlisting}

\newpage
\subsubsection{Lecturers}

The lecture table keeps record of details of lectures of various departments,
it is used to keeps track of their salaries as well.

\begin{lstlisting}[basicstyle=\small]
       Column        |          Type          | Nullable
---------------------+------------------------+----------
 lecturer_id         | character varying(15)  | not null
 name                | character varying(30)  |
 fathers_name        | character varying(30)  |
 spouse_name         | character varying(30)  |
 phone_num           | numeric(10,0)          |
 dob                 | date                   |
 email               | character varying(30)  |
 address             | character varying(255) |
 qualification       | character varying(50)  |
 subject_expertise   | character varying(50)  |
 department          | character varying(5)   |
 experience          | numeric(3,1)           |
 aadhar_number       | numeric(12,0)          |
 pan_number          | character varying(10)  |
 bank_account_number | numeric(15,0)          |
 pay_scale           | integer                |
 basic_pay           | integer                |
 gross_salary        | integer                |
 deduction           | integer                |
 net_salary          | integer                |
\end{lstlisting}

\subsubsection{Students leave}

This table is used to keep records of the requests made by students for leave.
As a student requests for leave it is inserted into the table but the leave is
not assigned immediately; instead it is initially marked as not assigned; the
HOD of the department to which the student belongs is then requested to approve
the leave for the students.

\begin{lstlisting}[basicstyle=\small]
   Column   |         Type           | Nullable | Default
------------+------------------------+----------+---------
 department | character varying(5)   | not null |
 usn        | character varying(15)  | not null |
 date       | date                   | not null |
 assigned   | boolean                |          | false
\end{lstlisting}

\newpage
\subsubsection{Lecturer leave}

This table is used to keep records of the requests made by lecturers for leave.
As a lecturer requests for leave it is inserted into the table but the leave is
not assigned immediately; instead it is initially marked as not assigned; the
HOD of the department to which the lecturer belongs to, is requested to approve
the leave for the lecturer.

\begin{lstlisting}[basicstyle=\small]
   Column    |         Type           | Nullable | Default
-------------+------------------------+----------+---------
 department  | character varying(5)   | not null |
 lecturer_id | character varying(15)  | not null |
 date        | date                   | not null |
 assigned    | boolean                |          | false
\end{lstlisting}

\subsubsection{Student attendance}

This table is used to keep track of attendance of the students per subject or
class. The lecture is the one who adds the attendance of the student.

\begin{lstlisting}[basicstyle=\small]
    Column    |         Type          | Nullable
--------------+-----------------------+----------
 usn          | character varying(15) | not null
 absent_date  | date                  | not null
 subject_code | character varying(10) | not null
\end{lstlisting}

\subsubsection{Lecturer attendance}

This table is used to keep track of the attendance of the lecturer. The head of
department to which the lecturer belongs to, is the only one who can adds this
information into the system.

\begin{lstlisting}[basicstyle=\small]
   Column    |         Type          | Nullable
-------------+-----------------------+----------
 lecturer_id | character varying(15) | not null
 absent_date | date                  | not null
\end{lstlisting}

\newpage
\subsubsection{Subjects}

This is the table that stores the all of the necessary information about the
different subjects that are across various departments in university. It holds
information about subjects, its name, its short code, who teaches the subject,
to which department the subject belongs to and the semester in which students
will study the subject.

\begin{lstlisting}[basicstyle=\small]
    Column    |         Type          | Nullable
--------------+-----------------------+----------
 subject_code | character varying(10) | not null
 subject_name | character varying(50) | not null
 taught_by    | character varying(15) | not null
 department   | character varying(5)  | not null
 semester     | numeric(1,0)          | not null
\end{lstlisting}

\subsubsection{Passwords}

The passwords table holds the ID and passwords of all the individuals that are
in a university, the students and all the lectures get their own unique ID and
a default password which is the same as their ID, all the individuals can choose
to change their passwords anytime they desire.

\begin{lstlisting}[basicstyle=\small]
  Column  |          Type          | Nullable
----------+------------------------+----------
 id       | character varying(15)  | not null
 password | character varying(255) |
\end{lstlisting}
