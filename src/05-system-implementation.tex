\section{SYSTEM IMPLEMENTATION}
The proposed application is in its entirety is very huge and so it is broken
down into smaller modules for better code management and ease of understanding,
and also for separation of concerns of different duties to different modules.

The different modules handle different jobs, and they are not concerned with
how other module performs; each individual module is completely independent from
each other and perform their own designated tasks.

\subsection{Courses and Subject Management Module}
As the main goal of every university is to provide a quality education, the
individuals should only focus on that, this module would ease the activity of
handling courses and subjects. The system will help the admin assign the
subjects per course, assign subjects to professors, decide on the syllabus of
individual subjects etc.

This module is responsible for handling all of the
course and subject related details. Only the admin is authorized to interact
with this module to make any changes. Without authorization, this module will
never be touch, as the request to make any changes would be stopped at the time
of authorization.

\subsection{Information Management Module}
The student and faculty information management cost a lot of time and effort to
do, plus the added costs when we need to know details of a particular student or
faculty, we have to sit and manually find the necessary information. While
this modules will do the storage and retrieval’s of the required details of
students and faculty information across various departments, the admin would
just fed the information to the system.

The information can only be fed into the system by the authorized admin, and the
information can only be accessed by the individuals who are authorized. Every
student and faculty is authorized to only access their own information and
unauthorized to access any other information about any other student or faculty
or any other staff members

\subsection{Attendance Monitoring Module}
This module enables handling of attendance of students and professors. There are
different ways in which this module is accessed such as: Both the students and
professors can access this module to check on their attendance, of course with
authorization.

The professors are the ones that are authorized to add attendance of students
that are enrolled into those subjects that the professor teaches.  The students
are only authorized to look at their attendance and not the attendance of other
students. When it comes to attendance of professors, only the head of department
to which they belong is authorized to add their attendance and every professor
is authorized to look at only their own attendance.

\subsection{Grading System Management Module}
Alike attendance, each professors is in charge of updating the marks of all the
students enrolled in his subjects. The professors can upload marks of internal
assessment tests and external assessment tests, this module is smart enough
that it understands that marks are being entered and update the “passed” status
of every student. The interesting part of this module is that only the
professors who teach are authorized to update the marks.

\subsection{Authorization Module}

The authorization module is the main part of this entire system to protect the
privacy and authenticity of information available within the system.
Authorization enables us to determine who is logging in and depending on that we
can provide for different UI for admin, students and professors. Apart from
providing different pages for UI, this module enables us to authorize who is
requesting for what information, and provide those information or appropriate
error messages depending on whether they are authorized or not.

We here make use of JWT (Json Web Tokens) for authorization, a token is created
when the user first logs in with their user ID and password, and sent to the
client which then stores it and send it along with every request that the client
makes on behalf of the user. This token is generated with the users password,
users ID and a secret key that only the server knows. On every request the
client makes this token is expected form it, if the token doesn't exists then
the request is terminated as unauthorized, if the token exists then it is first
decomposed back and user ID is known, this user ID is then used to determine if
the user is authorized to access some particular information that is requested
by the user, if he is then the authorization is completed and appropriate module
is handed over the control of what to do and what to return for a particular
request, if the if he is not authorized then the request is terminated there
itself with appropriate message.

\newpage
All of these different module help in gathering data from different sources and
collects them in a central place. Different individual are assigned and are
responsible to update different information about different aspects of the
university, thus eliminating the need of people to keep book records. The added
benefit of this is that students can log in and check their details, courses
and marks and know how well they are performing in different subjects.
Professors can log in and look at how well all the students performed in their
subjects and adjust their methodology to better suit the students.

These different modules interact with each other on the server and provide an
abstraction and hide the implementation details on how they work all while
providing us with an experience par from better. These different module of our
system work by allowing only authorized individuals to perform their part in a
bigger system.
